\documentclass[UTF8,a4paper]{ctexart}
%----------宏包区----------
\usepackage{fontspec}
\usepackage{ctex}
\usepackage{amsmath}
\usepackage{amssymb}
\usepackage{theorem}
\usepackage{bm}
\usepackage{geometry}
\usepackage{fancyhdr}
\usepackage{multicol}
\usepackage{footmisc}
\usepackage{longtable}
\usepackage{multirow}
\usepackage{listings}
\usepackage{hyperref}
\usepackage{float}
\usepackage{caption}
\usepackage{graphics}
\usepackage{graphicx}
\usepackage{xcolor}
\usepackage{tikz}
\usepackage{enumerate}
\usepackage{subfigure}
\usepackage{float}
\usepackage{threeparttable}
\usepackage{threeparttablex}
\usepackage{booktabs}
\usepackage{cite} 
\usepackage{hologo}
\numberwithin{table}{section}
\numberwithin{equation}{section}
\geometry{left=2.5cm,right=2.5cm,top=3.5cm,bottom=2.5cm} 
\usepackage{fontspec}
\setmainfont{Times New Roman}
%\setmainfont{Source Serif Pro}%Times New Roman
\usepackage{etoolbox}
\usepackage{fancyhdr}
\pagestyle{fancy}
\lhead{}
\rhead{}
\chead{\zihao{5} 大连理工大学本科毕业设计的\LaTeX 模板}
\hypersetup{
	colorlinks=true,
	linkcolor=black,
	citecolor=black,        
}
\usepackage{gbt7714}
\ctexset{
	contentsname = {\zihao{-3} {\textcolor{black}{\heiti 目\quad\quad 录}}},
	bibname = {\zihao{-4} {\textcolor{black}{参考文献}}}
}

\captionsetup[table]{labelsep=space}
\captionsetup[figure]{labelsep=space}
%space去掉点
%period加点
%不加space、period这两个就是冒号

\RequirePackage{titletoc}
\usepackage{titlesec}
\usepackage{titletoc}
\titlecontents{section}[1em]{\zihao{-4}}{\thecontentslabel ~~}{}{\hspace{.5em}\titlerule*{.}\contentspage}
\titlecontents{subsection}[2em]{\zihao{-4}}{\thecontentslabel ~~}{}{\hspace{.5em}\titlerule*{.}\contentspage}
\titlecontents{subsubsection}[3em]{\zihao{-4}}{\thecontentslabel ~~}{}{\hspace{.5em}\titlerule*{.}\contentspage}
%---------主代码------------
\begin{document}
	\begin{titlepage}
		\begin{center}
			~\\[1pt]
			\zihao{-1} {\bf{大连理工大学毕业设计(论文)}}
			~\\[12pt]
			\zihao{2} {\bf{大连理工大学毕业设计(论文)题目}}\\
			\zihao{3} {\bf{The Subject of Undergraduate Graduation Project (Thesis) of DUT}}
			~\\
			~\\
			~\\
			~\\
			~\\
			\zihao{-3} {学\ 院(系):\_\_\_\_\_\_\_\_\_\_\_\_\_\_\_\_\_\_\_\_\_\_\_}\\
			\zihao{-3} {专\qquad \quad  业:\_\_\_\_\_\_\_\_\_\_\_\_\_\_\_\_\_\_\_\_\_\_\_}\\
			\zihao{-3} {学\ 生\ 姓\ 名:\_\_\_\_\_\_\_\_\_\_\_\_\_\_\_\_\_\_\_\_\_\_\_}\\
			\zihao{-3} {学\qquad \quad 号:\_\_\_\_\_\_\_\_\_\_\_\_\_\_\_\_\_\_\_\_\_\_\_}\\
			\zihao{-3} {指\ 导\ 教\ 师:\_\_\_\_\_\_\_\_\_\_\_\_\_\_\_\_\_\_\_\_\_\_\_}\\
			\zihao{-3} {评\ 阅\ 教\ 师:\_\_\_\_\_\_\_\_\_\_\_\_\_\_\_\_\_\_\_\_\_\_\_}\\
			\zihao{-3} {完\ 成\ 日\ 期:\_\_\_\_\_\_\_\_\_\_\_\_\_\_\_\_\_\_\_\_\_\_\_}\\
			~\\
			~\\
			~\\
			~\\
			~\\
			~\\
			~\\
			{\zihao{-2} \xk{大连理工大学}}\\
			{\zihao{-4} Dalian University of Technology}
		\end{center}
	\newpage
	\end{titlepage}
	
	\newpage
	
	\section*{\fontsize{22}{33} {\heiti 原创性声明}}
	{\fontsize{15pt}{22.5pt} 本人郑重声明:本人所呈交的毕业设计(论文),是在指导老师的指导下独立进行研究所取得的成果。毕业设计(论文)中凡引用他人已经发表的成果、数据、观点等,均已明确注明出处。除文中已经注明引用的内容外,不包含任何其他个人或集体已经发表或撰写过的科研成果。对本文的研究成果做出重要贡献的个人和集体,均已在文中以明确方式标明。\\
	~\\
	本声明的法律责任由本人承担。\\
	~\\
	~\\
	作者签名: \qquad \qquad \qquad \qquad \qquad \qquad 日\quad 期:}
	\newpage
	\section*{\fontsize{22}{33} {\heiti 关于使用授权的声明}}
	{\fontsize{15pt}{22.5pt} 本人在指导老师的指导下所完成的毕业设计(论文)及相关的资料(包括图纸、试验记录、原始数据、实物照片、图片、录音带、设计手稿等),知识产权归属大连理工大学。本人完全了解大连理工大学有关保存、使用毕业设计(论文)的规定,本人授权大连理工大学可以将本毕业设计(论文)的全部或部分内容编入有关数据库进行检索,可以采用任何复制手段保存和汇编本毕业设计(论文)。如果发表相关成果,一定征得导师同意,且第一署名单位为大连理工大学。本人离校后使用毕业设计(论文)或发表直接相关学术论文或成果时,第一署名仍为大连理工大学。
	~\\
	~\\
	论文作者签名: \quad \quad \quad \quad \quad \quad 日\quad 期:\\
	指导老师签名: \quad \quad \quad \quad \quad \quad 日\quad 期:}
	\newpage
	%----------中文摘要-------------
	%------------------------------
	\addcontentsline{toc}{section}{\zihao{-4} {摘\quad\quad 要}} 
	\section*{\fontsize{15}{22.5} {\heiti 摘\quad\quad 要}}
	“摘要”是摘要部分的标题,不可省略。\par 
	标题“摘要”选用模板中的样式所定义的“标题1”,再居中;或者手动设置成字体:黑体,居中,字号:小三,1.5倍行距,段后11磅,段前为0。\par 
	摘要是毕业设计(论文)的缩影,文字要简练、明确。\par 
	内容要包括目的、方法、结果和结论。单位采用国际标准计量单位制,除特别情况外,数字一律用阿拉伯数码。文中不允许出现插图。重要的表格可以写入。\par 
	摘要正文选用模板中的样式所定义的“正文”,每段落首行缩进2个汉字;或者手动设置成每段落首行缩进2个汉字,字体:宋体,字号:小四,行距:多倍行距 1.25,间距:段前、段后均为0行,取消网格对齐选项。\par 
	摘要篇幅以一页为限,字数限500字以内。\par 
	摘要正文后,列出3-5个关键词。“关键词:”是关键词部分的引导,不可省略。关键词请尽量用《汉语主题词表》等词表提供的规范词。\par 
	关键词与摘要之间空一行。关键词词间用分号间隔,末尾不加标点,3-5个;黑体,小四,加粗。关键词整体字数限制在一行。\par 
	\quad \\
	{\bf{\heiti 关键词:写作规范;排版格式;毕业设计(论文)}}
	
	\newpage
	%---------英文摘要----------
	%------------------------------
	\section*{{\fontsize{15}{22.5} {The Subject of Undergraduate Graduation Project (Thesis) of DUT}}}
	\addcontentsline{toc}{section}{\zihao{-4} {Abstract}} 
	\section*{\fontsize{15}{22.5} Abstract}
	外文摘要要求用英文书写,内容应与“中文摘要”对应。使用第三人称,最好采用现在时态编写。\par 
	“Abstract”不可省略。标题“Abstract”选用模板中的样式所定义的“标题1”,再居中;或者手动设置成字体:Times New Roman,居中,字号:小三,多倍行距1.5倍行距,段后11磅,段前为0行。\par 
	标题“Abstract”上方是论文的英文题目,字体:Times New Roman,居中,字号:小三,行距:多倍行距 1.25,间距:段前、段后均为0行,取消网格对齐选项。\par 
	Abstract正文选用设置成每段落首行缩进2字,字体:Times New Roman,字号:小四,行距:多倍行距 1.25,间距:段前、段后均为0行,取消网格对齐选项。\par 
	Key words与摘要正文之间空一行。Key words与中文“关键词”一致。词间用分号间隔,末尾不加标点,3-5个;Times New Roman,小四,加粗。\par 
	\quad \\
	{\bf{Key words:Write Criterion;Typeset Format;Graduation Project (Thesis)} }
	
	\newpage
	\tableofcontents
	\newpage
	%------------------------
	%------------------------------
	\pagenumbering{arabic}
	\setcounter{page}{1}
	\addcontentsline{toc}{section}{\zihao{-4} {引\quad\quad 言}} 
	\section*{\fontsize{15}{22.5} {\heiti 引\quad\quad 言}}
	理工文科所有专业本科生的毕业设计(论文)都应有“引言”的内容。如果引言部分省略,该部分内容在正文中单独成章,标题改为文献综述,用足够的文字叙述。从引言开始,是正文的起始页,页码从1开始顺序编排。\par 
	针对做毕业设计:说明毕业设计的方案理解,阐述设计方法和设计依据,讨论对设计重点的理解和解决思路。\par 
	针对做毕业论文:说明论文的主题和选题的范围;对本论文研究主要范围内已有文献的评述;说明本论文所要解决的问题。建议与相关历史回顾、前人工作的文献评论、理论分析等相结合。\par 
	注意:是否如实引用前人结果反映的是学术道德问题,应明确写出同行相近的和已取得的成果,避免抄袭之嫌。注意不要与摘要内容雷同。\par 
	书写格式说明:\par 
	标题“引言”选用模板中的样式所定义的“引言”;或者手动设置成字体:黑体,居中,字号:小三,1.5倍行距,段后1行,段前为0行。\par
	引言的字数在3000字左右(毕业设计类引言可适当调整为800字左右)。引言正文选用模板中的样式所定义的“正文”,每段落首行缩进2字;或者手动设置成每段落首行缩进2字,宋体,小四,多倍行距 1.25,段前、段后均为0行,取消网格对齐选项。
	\newpage
	%------------------------------
	%------------------------------
	\CTEXsetup[format={\bfseries}]{section}
	%----------正文------------
	%------------------------------
	\section{\fontsize{15}{22.5} {\heiti 示例}} 
	\subsection{\fontsize{14}{21}{\heiti 图片示例}}
	\begin{figure}[h]
		\centering
		\caption{\ \ 这是一张图片的示例}
		\includegraphics[scale=0.4]{1}
	\end{figure}
	\subsection{\fontsize{14}{21}{\heiti 引用文献示例}}
	随便写一行来引用一个中文文献\cite{姬丽娜2017基于GPU的视频流人群实时计数}\par 
	随便写一行来引用一个英文文献\cite{Hubel1962Receptive}
	\newpage
	%------------结论----------------
	%--------------------------------
	\CTEXsetup[format={\centering}]{section}
	\addcontentsline{toc}{section}{\zihao{-4} {结\quad\quad 论(设计类为设计总结)}} 
	\section*{\fontsize{15}{22.5} {\heiti 结\quad\quad 论(设计类为设计总结)}}
	结论是理论分析和实验结果的逻辑发展,是整篇论文的归宿。结论是在理论分析、试验结果的基础上,经过分析、推理、判断、归纳的过程而形成的总观点。结论必须完整、准确、鲜明、并突出与前人不同的新见解。\par 
	书写格式说明:\par 
	标题“结论”选用模板中的样式所定义的“结论”,或者手动设置成字体:黑体,居中,字号:小三,1.5倍行距,段后1行,段前为0行。\par
	结论正文选用模板中的样式所定义的“正文”,每段落首行缩进2字;或者手动设置成每段落首行缩进2字,字体:宋体,字号:小四,行距:多倍行距 1.25,间距:段前、段后均为0行。\par 
	\newpage
	%-------------------------------------
	%----------------文献-----------------
	\renewcommand{\refname}{\fontsize{15}{22.5} \bf{\heiti 参\ 考\ 文\ 献}}
	\addcontentsline{toc}{section}{\zihao{-4} {参\ 考\ 文\ 献}} 
	\bibliographystyle{gbt7714-numerical}
	\bibliography{ref}
	%-----------附录---------------
	%------------------------------
	\newpage
	\addcontentsline{toc}{section}{\zihao{-4} {附录A \quad 附录内容名称}} 
	\section*{\fontsize{15}{22.5} {\heiti 附录A \quad 附录内容名称}}
	以下内容可放在附录之内:\par 
	(1) 正文内过于冗长的公式推导;\par 
	(2) 方便他人阅读所需的辅助性数学工具或表格;\par 
	(3) 重复性数据和图表;\par 
	(4) 论文使用的主要符号的意义和单位;\par 
	(5) 程序说明和程序全文;\par 
	(6) 调研报告;\par 
	(7) 翻译部分有关说明。\par 
	这部分内容可省略。如果省略,删掉此页。\par 
	书写格式说明:\par 
	标题“附录A 附录内容名称”选用模板中的样式所定义的“附录”;或者手动设置成字体:黑体,居中,字号:小三,1.5倍行距,段后1行,段前为0行。\par 
	附录正文选用模板中的样式所定义的“正文”,每段落首行缩进2字;或者手动设置成每段落首行缩进2字,字体:宋体,字号:小四,行距:多倍行距 1.25,间距:段前、段后均为0行。
	
	\newpage
	%------------修改记录----------
	%------------------------------
	\addcontentsline{toc}{section}{\zihao{-4} {修改记录}} 
	\section*{\fontsize{15}{22.5} {\heiti 修改记录}}
	修改是论文写作过程中不可或缺的重要步骤,是提高论文质量的有效环节。修改的过程其实就是“去伪存真”、去糟粕取精华使论文不断“升华”的过程。\par 
	以下内容要求放到毕业设计(论文)修改记录中:\par 
	(1) 毕业设计(论文)题目修改\par 
	{\bf{第一次修改记录:}}(没有可删除,后面记录依次递进)\par 
	原题目:\par 
	修稿后题目:\par 
	(2) 指导教师变更\par
	{\bf{第二次修改记录:}}(没有可删除,后面记录依次递进)\par
	原指导教师:******更改后指导教师:******\par
	(3) 校外毕业设计(论文)时间节点记录\par
	{\bf{第三次修改记录:}}(没有可删除,后面记录依次递进)\par
	本人于2019年1月申请到******大学做毕业设计(论文),指导教师为:******\par
	校内指导教师为:******。2019年*月*日回到学校。\par
	(4) 毕业设计(论文)内容重要修改记录\par
	包括:指导教师要求的重大修改,评阅教师要求的修改,答辩委员会提出的修改意见以及检测后的修改记录等。\par
	{\bf{第四次修改记录:}}(如实记录重要修改,不可省略)\par
	第5页2.1,{\bf{修改前:}}\par
	{\bf{修改后:}}\par
	{\bf{第五次修改记录:}}\par
	第8页表2.4表名,{\bf{修改前:}}\par
	{\bf{修改后:}}\par
	{\bf{第六次修改记录:}}\par
	(5) 毕业设计(论文)外文翻译修改记录\par
	(6) 毕业设计(论文)正式检测重复比\par
	修改记录正文选用模板中的样式所定义的“正文”,每段落首行缩进2字;字体:宋体,字号:小四,行距:多倍行距 1.25,间距:段前、段后均为0行。\par
	\qquad\qquad\qquad\qquad\qquad\qquad\qquad\qquad\qquad\qquad\qquad\qquad 记录人(签字):\par
	\qquad\qquad\qquad\qquad\qquad\qquad\qquad\qquad\qquad\qquad\qquad\quad 指导教师(签字):\par
	
	\newpage
	%----------致谢----------------
	%------------------------------
	\addcontentsline{toc}{section}{\zihao{-4} {致\quad\quad 谢}} 
	\section*{\fontsize{15}{22.5} {\heiti 致\quad\quad 谢}}
	毕业设计(论文)致谢中不得书写与毕业设计(论文)工作无关的人和事,对指导老师的致谢要实事求是。\par 
	对其他在本研究工作中提出建议和给予帮助的老师和同学,应在论文中做明确的说明并表示谢意。\par 
	这部分内容不可省略。\par 
	书写格式说明:\par 
	标题“致谢”选用模板中的样式所定义的“致谢”;或者手动设置成字体:黑体,居中,字号:小三,1.5倍行距,段后1行,段前为0行。\par 
	致谢正文选用模板中的样式所定义的“正文”,每段落首行缩进2字;或者手动设置成每段落首行缩进2字,字体:宋体,字号:小四,行距:多倍行距 1.25,间距:段前、段后均为0行。
	
\end{document}
