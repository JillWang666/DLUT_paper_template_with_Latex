%——————————————————————————————————————————————————————————————————
%———————————————————————————————————————————————————————————————————
% |作者:王海桁                                                     |
% |                            -注意-                              | 
% | 	!所有的行距都需要用spacing环境自定义,具体数值可以参看文档!   |
% |                   !禁止使用\bf命令进行加粗!                    |
% |          !目前目录是需要自己修改格式的,具体看文档!              |
% |                  !目前的页码样式是错的,还在改!                 |
% |                                                                |
%———————————————————————————————————————————————————————————————————
%——————————————————————————————————————————————————————————————————
%                         ______________________
%                         |                    |
%                         |    图片的插入方式   |    
%                         |                    |
%                         ———————————————————————
%
%
%                                  ————————————》 图X.X  XXXXXXXX
%                                 |        _________________________________      
%			          |      |                                 |
%			          |      |                                 |
%			          |      |   ██╔══██╗██║   ██║██╔════╝     |
%                       对应位置   |      |   ██████╔╝██║   ██║██║  ███╗    |
%                                 |      |   ██╔══██╗██║   ██║██║   ██║    |
%                                 |      |   ██████╔╝╚██████╔╝╚██████╔╝    |
%                                 |      |   ╚═════╝  ╚═════╝  ╚═════╝     |                          
%		                  |      |                                 |
%                                 |      |                                 |
%                                 |      ——————————————————————————————————-
%                                 |                     ^
%                                 |                   ^ | ^
%		\begin{figure}[h]                    ^  |  ^          
%		    \centering                          |  对应位置
%		    \caption{   XXXXXXXXX   }           |
%		    \includegraphics[scale=XX]{XX} ——————
%		\end{figure}
%——————————————————————————————————————————————————————————————————
%——————————————————————————————————————————————————————————————————
\documentclass[UTF8,a4paper,zihao=-4,AutoFakeBold,scheme=chinese]{ctexart}
%———————————————————————————————————————————————————————————————————
% |                                                                |
% |                            -宏包区-                             | 
% |                     !不推荐更改宏包的顺序!                     |
%———————————————————————————————————————————————————————————————————
\usepackage{xcolor}								%花里胡哨的颜色  \color{Peach} !此包最好在宏包区置顶!
\usepackage{pmat}  								%zmatlab代码高亮环境
\usepackage{fontspec}							%局部行距  \begin&end{spacing}{行距1.6}
\usepackage{setspace}							%全局行距 \linespread{行距1.6} 或局部加空行 
\usepackage{ctex}								%ctex环境包,带字体\heiti
\usepackage{amsmath}							%数学环境
\usepackage{amssymb}							%数学环境
%————————————————————————————————————————————————————————————————————————————————————————————
\usepackage{theorem}                            %数学定理环境					|					
%										 	     |
%       大环境I————————》\section{theorem}		   \section{Proof}《——————大环境II        |
%	小环境I————————》	\begin{theorem}		   \begin{proof} 			  |	
%				This is a theorem.        This is a proof		     |
%			    \end{theorem}	      \end{proof}			     |
%											     |
%	  小环境II————————》\begin{lemma}   						  |
%				This is a theorem.					     |
%			    \end{lemma}							     |
%————————————————————————————————————————————————————————————————————————————————————————————
%\usepackage{ntheorem}							%更精细的数学定理环境	!和上面的二选一!			    
\usepackage{bm}								%强制加粗,\bm{内容} 
%——————————————————————————全局页边距——————————————————————————
\usepackage{geometry}							%页边距       |
\geometry{left=2.5cm,right=2.5cm,top=3.5cm,bottom=2.5cm}    %|
%—————————————————————————————————————————————————————————————
%—————————————————————————————————————————————————————————————
%———————————————————————————————————————————————————————————————————--
%                                                                    |
%                               -页眉-                               | 
%                                                                    |
\usepackage{fancyhdr}					            %|
\pagestyle{fancy}						    %|
\lhead{}							    %|
\rhead{}							    %|
\chead{\zihao{5} The Subject of Undergraduate Graduation Project (Thesis) of DUT}																				    %|
%———————————————————————————————————————————————————————————————————--
%\usepackage{multicol}						%双栏排版
\usepackage{footmisc}						%脚注  \footnote{内容}
\usepackage{longtable}						%更长的表格
\usepackage{multirow}						%不规则表格 和 格内合并
\usepackage{listings}						%代码环境 \begin&end{lstings} \lstset{language=c++}
%———————————————————————————————————————————————————————————————————————————————————
\usepackage[hidelinks]{hyperref}				%花里胡哨的文字高亮 !会显示在PDF上!绝对不要改变这个包的位置!  
%\hypersetup{																		
%	colorlinks=true,					%超链接								 
%	linkcolor=cyan,						%公式								 
\numberwithin{equation}{section}				%公式序号							 
%	filecolor=blue,      					%文件								
%	urlcolor=red,						%网页								
%	citecolor=green,					文献								
%}																					
%\href{网址}{起个名字显示在文本上}													  
%———————————————————————————————————————————————————————————————————————————————————
\usepackage{float}						%合适的结构体位置
%——————————————————————————————————————————题注———————————————————————————————————————————————————
\usepackage{caption}						%题注,给图片表格等插入环境命名的 \caption{内容}	
\captionsetup[table]{labelsep=space}				%表格题注										  		
\captionsetup[figure]{labelsep=space}				%图片题注										  
%—————————————————————————————————————————————————————————————————————————————————————————————————
\usepackage{graphics}						%图片环境I
\usepackage{graphicx}						%图片环境II
%\usepackage{tikz}						%tex内置的画图!非常强也非常难!  https://zhuanlan.zhihu.com/p/142405882
\usepackage{enumerate}						%序号 \item
\usepackage{subfigure}						%多图并排	
%——————————————————————————————表格——————————————————————————————————						
\usepackage{threeparttable}					%三线表I			
\usepackage{threeparttablex}					%三线表II			
\usepackage{booktabs}						%三线表III			
\numberwithin{table}{section}					%表格序号			
%————————————————————————————————————————————————————————————————————
\usepackage{cite} 						%引用	\cite{}
%\usepackage{hologo}						%可以用来输入一些Latex的logo   \Hologo{name}  
\usepackage{etoolbox}						%更强的全局命令赋值
%——————————————————————————————目录——————————————————————————————————————————————————————————————————————————————————————
\RequirePackage{titletoc}																								
\usepackage{titlesec}																									
\usepackage{titletoc}																									
\titlecontents{section}[1em]{\zihao{-4}}{\thecontentslabel ~~}{}{\hspace{.5em}\titlerule*{.}\contentspage}				
\titlecontents{subsection}[2em]{\zihao{-4}}{\thecontentslabel ~~}{}{\hspace{.5em}\titlerule*{.}\contentspage}			
\titlecontents{subsubsection}[3em]{\zihao{-4}}{\thecontentslabel ~~}{}{\hspace{.5em}\titlerule*{.}\contentspage}		
\usepackage{gbt7714}						%文献格式							
\ctexset{																												
	contentsname = {\zihao{-3} {\textcolor{black}{\heiti 目\quad\quad 录}}},											
	bibname = {\zihao{-4} {\textcolor{black}{参考文献}}}														
}																			
%———————————————————————————————————————————————————————————————————————————————————————————————————————————————————————

%—————————————————————————————————————————————————————————————————————
%                                                               	 
%                             -定义字体-                               
%                                                                	 
\setmainfont{Times New Roman}					%全局英文		 
%																	
\setCJKfamilyfont{hwxk}{华文行楷}									
\newcommand*{\xk}{\CJKfamily{hwxk}}									
\newcommand*{\bxk}{\textbf{\CJKfamily{hwxk}}}						
%																	
\setCJKfamilyfont{xh}{华文细黑}															
\newcommand*{\xh}{\CJKfamily{xh}}									
%																	
\setCJKfamilyfont{st}{宋体}											
\newcommand*{\st}{\CJKfamily{st}}									
\newcommand*{\bst}{\textbf{\CJKfamily{st}}}							
%—————————————————————————————————————————————————————————————————————

%———————————————————————————————————————————————————————————————————
% |                                                                |
% |                             主代码                             | 
% |                                                                |
%———————————————————————————————————————————————————————————————————
\begin{document}
\renewcommand{\figurename}{\zihao{5} \songti 图}	
\renewcommand{\tablename}{\zihao{5} \songti 表}
	%———————————————————————————————————————————————————————————————————
	% |                                                                |
	% |                           题目 + 目录                          | 
	% |                                                                |
	%———————————————————————————————————————————————————————————————————
	\begin{titlepage}
		\begin{center}
			~\\[8pt]
			\zihao{-1} {\songti\textbf{大连理工大学本科毕业设计(论文)}}
			~\\[35pt]
			\textbf{\zihao{2}{\xh{大连理工大学本科毕业设计(论文)题目}}}
			~\\[15pt]
		\end{center}
		%!英文标题,如果你的标题是一行显示的开的请把这句话加进上下的center环境里
		\textbf{\zihao{3}{The Subject of Undergraduate Graduation Project (Thesis) of DUT}}
		%!英文标题,如果你的标题是一行显示!不开!的请移出上下的center环境并适当调节至居中
		\begin{center}
			~\\[120pt]
			\begin{spacing}{1.7}
				\zihao{-3} {\songti 学\ 院(系):\_\_\_\_\_\_\_\_\_\_\_\_\_\_\_\_\_\_\_\_\_\_\_}\\
				\zihao{-3} {\songti 专\qquad \quad  业:\_\_\_\_\_\_\_\_\_\_\_\_\_\_\_\_\_\_\_\_\_\_\_}\\
				\zihao{-3} {\songti 学\ 生\ 姓\ 名:\_\_\_\_\_\_\_\_\_\_\_\_\_\_\_\_\_\_\_\_\_\_\_}\\
				\zihao{-3} {\songti 学\qquad \quad 号:\_\_\_\_\_\_\_\_\_\_\_\_\_\_\_\_\_\_\_\_\_\_\_}\\
				\zihao{-3} {\songti 指\ 导\ 教\ 师:\_\_\_\_\_\_\_\_\_\_\_\_\_\_\_\_\_\_\_\_\_\_\_}\\
				\zihao{-3} {\songti 评\ 阅\ 教\ 师:\_\_\_\_\_\_\_\_\_\_\_\_\_\_\_\_\_\_\_\_\_\_\_}\\
				\zihao{-3} {\songti 完\ 成\ 日\ 期:\_\_\_\_\_\_\_\_\_\_\_\_\_\_\_\_\_\_\_\_\_\_\_}\\
			\end{spacing}
			~\\[70pt]
			{\zihao{-2} \xk{大连理工大学}}\\
			{\zihao{-4} Dalian University of Technology}
		\end{center}
	\newpage
	\end{titlepage}
	
	\newpage
	%———————————————————————————————————————————————————————————————————
	% |                                                                |
	% |                           原创性声明                            | 
	% |                                                                |
	%———————————————————————————————————————————————————————————————————
	\begin{spacing}{2}
		\section*{\textbf{\zihao{2}{\xh{原创性声明}}}}
	\end{spacing}
	\begin{spacing}{1.7}
	{\zihao{-3} \songti   本人郑重声明:本人所呈交的毕业设计(论文),是在指导老师的指导下独立进行研究所取得的成果。
	毕业设计(论文)中凡引用他人已经发表的成果、数据、观点等,均已明确注明出处。除文中已经注明引用的内容外,不
	包含任何其他个人或集体已经发表或撰写过的科研成果。对本文的研究成果做出重要贡献的个人和集体,均已在文中以明
	确方式标明。\\
	~\\
	本声明的法律责任由本人承担。\\
	~\\
	~\\
	作者签名: \qquad \qquad \qquad \qquad \quad  日\quad 期:}
	\end{spacing}
	\newpage
	%———————————————————————————————————————————————————————————————————
	% |                                                                |
	% |                           授权声明页                            | 
	% |                                                                |
	%———————————————————————————————————————————————————————————————————
	\begin{spacing}{2}
		\section*{\textbf{\zihao{2}{\xh{关于使用授权的声明}}}}
	\end{spacing}
	\begin{spacing}{1.7}
		{\zihao{-3} \songti 本人在指导老师的指导下所完成的毕业设计(论文)及相关的资料(包括图纸、试验记录
	、原始数据、实物照片、图片、录音带、设计手稿等),知识产权归属大连理工大学。本人完全了解大连理工大学有关保
	存、使用毕业设计(论文)的规定,本人授权大连理工大学可以将本毕业设计(论文)的全部或部分内容编入有关数据
	库进行检索,可以采用任何复制手段保存和汇编本毕业设计(论文)。如果发表相关成果,一定征得导师同意,且第一
	署名单位为大连理工大学。本人离校后使用毕业设计(论文)或发表直接相关学术论文或成果时,第一署名仍为大连理
	工大学。
	~\\
	~\\
	~\\
	论文作者签名: \quad \quad \quad \qquad \qquad \quad 日\quad 期:\\
	指导老师签名: \quad \quad \quad \qquad \qquad \quad 日\quad 期:}
	\end{spacing}
	\newpage
	
	%———————————————————————————————————————————————————————————————————
	% |                                                                |
	% |                            中文摘要                             | 
	% |                                                                |
	%———————————————————————————————————————————————————————————————————
	\pagenumbering{Roman}
	\addcontentsline{toc}{section}{\zihao{-4} {摘\quad\quad 要}} 
	\begin{spacing}{1.7}
	\section*{\zihao{-3} {\heiti 摘\quad\quad 要}}
	\end{spacing}
	\begin{spacing}{1.4}
	{\zihao{-4}\songti “摘要”是摘要部分的标题,不可省略。\par 
	标题“摘要”选用模板中的样式所定义的“标题1”,再居中;或者手动设置成字体:黑体,居中,字号:小三,1.5倍行距,段后11磅,段前为0。\par 
	摘要是毕业设计(论文)的缩影,文字要简练、明确。\par 
	内容要包括目的、方法、结果和结论。单位采用国际标准计量单位制,除特别情况外,数字一律用阿拉伯数码。文中不允许出现插图。重要的表格可以写入。\par 
	摘要正文选用模板中的样式所定义的“正文”,每段落首行缩进2个汉字;或者手动设置成每段落首行缩进2个汉字,字体:宋体,字号:小四,行距:多倍行距 1.25,间距:段前、段后均为0行,取消网格对齐选项。\par 
	摘要篇幅以一页为限,字数限500字以内。\par 
	摘要正文后,列出3-5个关键词。“关键词:”是关键词部分的引导,不可省略。关键词请尽量用《汉语主题词表》等词表提供的规范词。\par 
	关键词与摘要之间空一行。关键词词间用分号间隔,末尾不加标点,3-5个;黑体,小四,加粗。关键词整体字数限制在一行。\par }
	\quad \\
	{\textbf{\zihao{-4} \heiti 关键词:写作规范;排版格式;毕业设计(论文)}}
	\end{spacing}
	\newpage
	%———————————————————————————————————————————————————————————————————
	% |                                                                |
	% |                            英文摘要                             | 
	% |                                                                |
	%———————————————————————————————————————————————————————————————————
	\section*{\textbf{\zihao{-3} {The Subject of Undergraduate Graduation Project (Thesis) of DUT}}}
	\addcontentsline{toc}{section}{\zihao{-4} {Abstract}} 
	\begin{spacing}{1.6}
		\section*{\zihao{-3} {Abstract}}
	\end{spacing}
	{\zihao{-4}\songti 外文摘要要求用英文书写,内容应与“中文摘要”对应。使用第三人称,最好采用现在时态编写。\par 
	“Abstract”不可省略。标题“Abstract”选用模板中的样式所定义的“标题1”,再居中;或者手动设置成字体:Times New Roman,居中,字号:小三,多倍行距1.5倍行距,段后11磅,段前为0行。\par 
	标题“Abstract”上方是论文的英文题目,字体:Times New Roman,居中,字号:小三,行距:多倍行距 1.25,间距:段前、段后均为0行,取消网格对齐选项。\par 
	Abstract正文选用设置成每段落首行缩进2字,字体:Times New Roman,字号:小四,行距:多倍行距 1.25,间距:段前、段后均为0行,取消网格对齐选项。\par 
	Key words与摘要正文之间空一行。Key words与中文“关键词”一致。词间用分号间隔,末尾不加标点,3-5个;Times New Roman,小四,加粗。\par 
	\quad \\
	{\textbf{\zihao{-4} Key words:Write Criterion;Typeset Format;Graduation Project (Thesis)} }}
	
	\newpage
	\tableofcontents
	\newpage
	%———————————————————————————————————————————————————————————————————
	% |                                                                |
	% |                             引言                               | 
	% |                                                                |
	%———————————————————————————————————————————————————————————————————
	\pagenumbering{arabic}
	\setcounter{page}{1}
	\addcontentsline{toc}{section}{\zihao{-4} {引\quad\quad 言}} 
	\begin{spacing}{1.7}
		\section*{\fontsize{15}{22.5} {\heiti 引\quad\quad 言}}
	\end{spacing}
	\begin{spacing}{1.4}
	{\zihao{-4}\songti 理工文科所有专业本科生的毕业设计(论文)都应有“引言”的内容。如果引言部分省略,该部分内容在正文中单独成章,标题改为文献综述,用足够的文字叙述。从引言开始,是正文的起始页,页码从1开始顺序编排。\par 
	针对做毕业设计:说明毕业设计的方案理解,阐述设计方法和设计依据,讨论对设计重点的理解和解决思路。\par 
	针对做毕业论文:说明论文的主题和选题的范围;对本论文研究主要范围内已有文献的评述;说明本论文所要解决的问题。建议与相关历史回顾、前人工作的文献评论、理论分析等相结合。\par 
	注意:是否如实引用前人结果反映的是学术道德问题,应明确写出同行相近的和已取得的成果,避免抄袭之嫌。注意不要与摘要内容雷同。\par 
	书写格式说明:\par 
	标题“引言”选用模板中的样式所定义的“引言”;或者手动设置成字体:黑体,居中,字号:小三,1.5倍行距,段后1行,段前为0行。\par
	引言的字数在3000字左右(毕业设计类引言可适当调整为800字左右)。引言正文选用模板中的样式所定义的“正文”,每段落首行缩进2字;或者手动设置成每段落首行缩进2字,宋体,小四,多倍行距 1.25,段前、段后均为0行,取消网格对齐选项。}
	\end{spacing}
	\newpage
	%——————————————————————————————————————————————————————————————————
	\CTEXsetup[format={\bfseries}]{section}
	%———————————————————————————————————————————————————————————————————
	% |                                                                |
	% |                             XXXX                               | 
	% |                                                                |
	%———————————————————————————————————————————————————————————————————
	\begin{spacing}{1.7}
		\section{\zihao{-3} {\heiti 大题目}} 
	\end{spacing}
	\begin{spacing}{1.3}
		\subsection{\zihao{4} {\heiti 小题目}}
	\end{spacing}
	\begin{spacing}{1.3}
		{\zihao{-4}\songti 正文。}
	\end{spacing}
	%———————————————————————————————————————————————————————————————————
	% |                                                                |
	% |                              结论                              | 
	% |                                                                |
	%———————————————————————————————————————————————————————————————————
	\CTEXsetup[format={\centering}]{section}
	\addcontentsline{toc}{section}{\zihao{-4} {结\quad\quad 论(设计类为设计总结)}} 
	\section*{\fontsize{15}{22.5} {\heiti 结\quad\quad 论(设计类为设计总结)}}
	结论是理论分析和实验结果的逻辑发展,是整篇论文的归宿。结论是在理论分析、试验结果的基础上,经过分析、推理、判断、归纳的过程而形成的总观点。结论必须完整、准确、鲜明、并突出与前人不同的新见解。\par 
	书写格式说明:\par 
	标题“结论”选用模板中的样式所定义的“结论”,或者手动设置成字体:黑体,居中,字号:小三,1.5倍行距,段后1行,段前为0行。\par
	结论正文选用模板中的样式所定义的“正文”,每段落首行缩进2字;或者手动设置成每段落首行缩进2字,字体:宋体,字号:小四,行距:多倍行距 1.25,间距:段前、段后均为0行。\par 
	\newpage
	%——————————————————————————————————————————————————————————————————
	% |                                                                |
	% |                           参考文献                              | 
	% |                                                                |
	%———————————————————————————————————————————————————————————————————
	\renewcommand{\refname}{\fontsize{15}{22.5} \bf{\heiti 参\ 考\ 文\ 献}}
	\addcontentsline{toc}{section}{\zihao{-4} {参\ \ 考\ \ 文\ \ 献}} 
	\bibliographystyle{gbt7714-numerical}
	\bibliography{ref}
	\newpage
	%———————————————————————————————————————————————————————————————————
	% |                                                                |
	% |                              附录                              | 
	% |                                                                |
	%———————————————————————————————————————————————————————————————————
	\addcontentsline{toc}{section}{\zihao{-4} {附录A \quad 附录内容名称}} 
	\section*{\fontsize{15}{22.5} {\heiti 附录A\quad 附录内容名称}}
	以下内容可放在附录之内:\par 
	(1) 正文内过于冗长的公式推导;\par 
	(2) 方便他人阅读所需的辅助性数学工具或表格;\par 
	(3) 重复性数据和图表;\par 
	(4) 论文使用的主要符号的意义和单位;\par 
	(5) 程序说明和程序全文;\par 
	(6) 调研报告;\par 
	(7) 翻译部分有关说明。\par 
	这部分内容可省略。如果省略,删掉此页。\par 
	书写格式说明:\par 
	标题“附录A 附录内容名称”选用模板中的样式所定义的“附录”;或者手动设置成字体:黑体,居中,字号:小三,1.5倍行距,段后1行,段前为0行。\par 
	附录正文选用模板中的样式所定义的“正文”,每段落首行缩进2字;或者手动设置成每段落首行缩进2字,字体:宋体,字号:小四,行距:多倍行距 1.25,间距:段前、段后均为0行。
	
	\newpage
	%———————————————————————————————————————————————————————————————————
	% |                                                                |
	% |                             修改记录                            | 
	% |                                                                |
	%———————————————————————————————————————————————————————————————————
	\addcontentsline{toc}{section}{\zihao{-4} {修改记录}} 
	\section*{\fontsize{15}{22.5} {\heiti 修改记录}}
	修改是论文写作过程中不可或缺的重要步骤,是提高论文质量的有效环节。修改的过程其实就是“去伪存真”、去糟粕取精华使论文不断“升华”的过程。\par 
	以下内容要求放到毕业设计(论文)修改记录中:\par 
	(1) 毕业设计(论文)题目修改\par 
	{\bf{第一次修改记录:}}(没有可删除,后面记录依次递进)\par 
	原题目:\par 
	修稿后题目:\par 
	(2) 指导教师变更\par
	{\bf{第二次修改记录:}}(没有可删除,后面记录依次递进)\par
	原指导教师:******更改后指导教师:******\par
	(3) 校外毕业设计(论文)时间节点记录\par
	{\bf{第三次修改记录:}}(没有可删除,后面记录依次递进)\par
	本人于2019年1月申请到******大学做毕业设计(论文),指导教师为:******\par
	校内指导教师为:******。2019年*月*日回到学校。\par
	(4) 毕业设计(论文)内容重要修改记录\par
	包括:指导教师要求的重大修改,评阅教师要求的修改,答辩委员会提出的修改意见以及检测后的修改记录等。\par
	{\bf{第四次修改记录:}}(如实记录重要修改,不可省略)\par
	第5页2.1,{\bf{修改前:}}\par
	{\bf{修改后:}}\par
	{\bf{第五次修改记录:}}\par
	第8页表2.4表名,{\bf{修改前:}}\par
	{\bf{修改后:}}\par
	{\bf{第六次修改记录:}}\par
	(5) 毕业设计(论文)外文翻译修改记录\par
	(6) 毕业设计(论文)正式检测重复比\par
	修改记录正文选用模板中的样式所定义的“正文”,每段落首行缩进2字;字体:宋体,字号:小四,行距:多倍行距 1.25,间距:段前、段后均为0行。\par
	\qquad\qquad\qquad\qquad\qquad\qquad\qquad\qquad\qquad\qquad\qquad\qquad 记录人(签字):\par
	\qquad\qquad\qquad\qquad\qquad\qquad\qquad\qquad\qquad\qquad\qquad\quad 指导教师(签字):\par
	
	\newpage
	%———————————————————————————————————————————————————————————————————
	% |                                                                |
	% |                              致谢                              | 
	% |                                                                |
	%———————————————————————————————————————————————————————————————————
	\addcontentsline{toc}{section}{\zihao{-4} {致\quad\quad 谢}} 
	\section*{\fontsize{15}{22.5} {\heiti 致\quad\quad 谢}}
	毕业设计(论文)致谢中不得书写与毕业设计(论文)工作无关的人和事,对指导老师的致谢要实事求是。\par 
	对其他在本研究工作中提出建议和给予帮助的老师和同学,应在论文中做明确的说明并表示谢意。\par 
	这部分内容不可省略。\par 
	书写格式说明:\par 
	标题“致谢”选用模板中的样式所定义的“致谢”;或者手动设置成字体:黑体,居中,字号:小三,1.5倍行距,段后1行,段前为0行。\par 
	致谢正文选用模板中的样式所定义的“正文”,每段落首行缩进2字;或者手动设置成每段落首行缩进2字,字体:宋体,字号:小四,行距:多倍行距 1.25,间距:段前、段后均为0行。
	
\end{document}
